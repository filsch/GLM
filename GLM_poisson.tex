\documentclass{article}
\topmargin -15mm
\textheight 24truecm   
\textwidth 16truecm    
\oddsidemargin 5mm
\evensidemargin 5mm   
\setlength\parskip{10pt}
\pagestyle{empty}          
\usepackage{boxedminipage}
\usepackage{amsfonts}
\usepackage{amsmath} 
\usepackage{amssymb}
\usepackage{amsthm}
\usepackage{t1enc}
\usepackage[utf8]{inputenc}
\usepackage{amssymb}
\usepackage{tikz}
\usepackage{commath}
\usepackage{listings}
\usepackage{verbatim}
\usepackage[margin=1in]{geometry}
\usepackage{caption}
\usepackage{subcaption}
\newcommand*\diff{\mathop{}\!\mathrm{d}}
\renewcommand{\thesubsection}{\thesection.\alph{subsection}}

\title{Assignment II - Generalized Linear Models}
\author{Filip Schjerven, Håkon Verås}
\date{\today}

\begin{document}

\maketitle
\begin{abstract}
In this project we have created a R-package called myglm and used it to discuss and analyze the salary-dataset. This involves implementing the print-, anova-, summary- and the myglm(...)-functions, similarly to the lm-package.
\end{abstract}
\newpage

\section{}

\subsection{}
One usually consider an offset when are dealing with an factor with \textit{known} slope. In our case we are studying the mortality rate due to lung cancer. It is then natural to consider the population-factor as an offset; If the population increases by a factor $a$, then one would obviously also expect that the number of deaths also increase by a factor $a$.

\subsection{}
We have that
\begin{equation*}
Y_{i} \sim \text{Po}(\mu_{i}),
\end{equation*}
so the Likelihood-function of $\mu_{i}$ is
\begin{equation*}
	L(y_{1},...,y_{n}|\mu_{i}) = \prod_{j=1}^{n} P(Y_{i} = y_{j}) = \prod_{j=1}^{n} \frac{e^{-\mu_{i}}\mu_{i}^{y_{j}}}{y_{j}!} = \frac{e^{-n\mu_{i}} \mu_{i}^{\sum_{j=1}^{n} y_{j}} }{\prod_{j=1}^{n} y_{j}!}.
\end{equation*}
Since we are modelling the log-mortality, we consider a generalized linear model with link log. We therefore have that
\begin{equation*}
\text{log}(\mu_{i}) = \text{offset}_{i} + \nu_{i} = \text{offset}_{i} + \boldsymbol{x_{i} \beta},
\end{equation*}
and
\begin{equation*}
\mu_{i} = \text{exp}(\text{offset}_{i} + \boldsymbol{x_{i} \beta}).
\end{equation*}
It follows that
\begin{equation*}
L(y_{1},...,y_{n}|\boldsymbol{\beta}) = \frac{\text{exp}\left(-n \cdot \text{exp}(\text{offset}_{i} + \boldsymbol{x_{i}\beta}) + \left(\text{offset}_{i} + \boldsymbol{x_{i}\beta}\right) \sum_{j=1}^{n}y_{j} \right)}{\prod_{j=1}^{n} y_{j}!}.
\end{equation*}


\begin{comment}
Further, the log-likelihood function can also be derived:
\begin{equation*}
\text{log}(L) = -n\mu_{i} + \left(\sum_{j=1}^{n} y_{j}\right)\text{log} (\mu_{i}) - \sum_{j = 1}^{n} \text{log}(y_{j}!)
\end{equation*}
\begin{equation*}
= -n\mu_{i} + \left(\sum_{j=1}^{n} y_{j}\right)(\text{offset}_{i} + \nu_{i}) - \sum_{j = 1}^{n} \text{log}(y_{j}!)
\end{equation*}
\begin{equation*}
= -n\mu_{i} + \left(\sum_{j=1}^{n} y_{j}\right)(\text{offset}_{i} + \boldsymbol{x_{i} \beta}) - \sum_{j = 1}^{n} \text{log}(y_{j}!)
\end{equation*}
Resonnementet og påfølgende likninger er jeg litt usikre på.
\begin{equation*}
\nu_{i} = \boldsymbol{x_{i} \beta}
\end{equation*}
\begin{equation*}
\text{log}(\mu_{i}) = \text{offset}_{i} + \nu_{i}
\end{equation*}
\begin{equation*}
\text{log}(\mu_{i}) = \boldsymbol{x_{i} \beta}
\end{equation*}
\end{comment}
\section{}
\subsection{}
\end{document}