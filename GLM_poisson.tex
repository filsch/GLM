\documentclass{article}
\topmargin -15mm
\textheight 24truecm   
\textwidth 16truecm    
\oddsidemargin 5mm
\evensidemargin 5mm   
\setlength\parskip{10pt}
\pagestyle{empty}          
\usepackage{boxedminipage}
\usepackage{amsfonts}
\usepackage{amsmath} 
\usepackage{amssymb}
\usepackage{amsthm}
\usepackage{t1enc}
\usepackage[utf8]{inputenc}
\usepackage{amssymb}
\usepackage{tikz}
\usepackage{commath}
\usepackage{listings}
\usepackage{verbatim}
\usepackage[margin=1in]{geometry}
\usepackage{caption}
\usepackage{subcaption}
\newcommand*\diff{\mathop{}\!\mathrm{d}}
\renewcommand{\thesubsection}{\thesection.\alph{subsection}}

\title{Assignment II - Generalized Linear Models}
\author{Filip Schjerven, Håkon Verås}
\date{\today}

\begin{document}

\maketitle
\begin{abstract}
In this project we have created a R-package called myglm and used it to discuss and analyze the salary-dataset. This involves implementing the print-, anova-, summary- and the myglm(...)-functions, similarly to the lm-package.
\end{abstract}
\newpage

\section{}

\subsection{}
The factor "age" differs from the other factors in the way that is a grouped binary factor and has a natural increase in mortality rate. The coefficients in "age" is not measured by the model, but the exposure for each level is considered beforehand.

\subsection{}
We have that
\begin{equation*}
Y_{i} \sim \text{Po}(\mu_{i}),
\end{equation*}
so we have that
\begin{equation*}
	L(y_{1},...,y_{n}|\mu_{i}) = \prod_{j=1}^{n} P(Y_{i} = y_{j}) = \prod_{j=1}^{n} \frac{e^{-\mu_{i}}\mu_{i}^{y_{j}}}{y_{j}!} = \frac{e^{-n\mu_{i}} \mu_{i}^{\sum_{j=1}^{n} y_{j}} }{\prod_{j=1}^{n} y_{j}!}.
\end{equation*}
Then
\begin{equation*}
\text{log}(L) = -n\mu_{i} + \left(\sum_{j=1}^{n} y_{j}\right)\text{log} (\mu_{i}) - \sum_{j = 1}^{n} \text{log}(y_{j}!)
\end{equation*}
\begin{equation*}
= -n\mu_{i} + \left(\sum_{j=1}^{n} y_{j}\right)(\text{offset}_{i} + \nu_{i}) - \sum_{j = 1}^{n} \text{log}(y_{j}!)
\end{equation*}
\begin{equation*}
= -n\mu_{i} + \left(\sum_{j=1}^{n} y_{j}\right)(\text{offset}_{i} + \boldsymbol{x_{i} \beta}) - \sum_{j = 1}^{n} \text{log}(y_{j}!)
\end{equation*}
Resonnementet og påfølgende likninger er jeg litt usikre på.
\begin{equation*}
\nu_{i} = \boldsymbol{x_{i} \beta}
\end{equation*}
\begin{equation*}
\text{log}(\mu_{i}) = \text{offset}_{i} + \nu_{i}
\end{equation*}
\begin{equation*}
\text{log}(\mu_{i}) = \boldsymbol{x_{i} \beta}
\end{equation*}

\begin{equation}

\section{}
\subsection{}

\end{document}